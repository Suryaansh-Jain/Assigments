%
% Welcome to Overleaf --- just edit your LaTeX on the left,
% and we'll compile it for you on the right. If you open the
% 'Share' menu, you can invite other users to edit at the same
% time. See www.overleaf.com/learn for more info. Enjoy!
%
%%%%%%%%%%%%%%%%%%%%%%%%%%%%%%%%%%%%%%%%%%%%%%%%%%%%%%%%%%%%%%%


% Inbuilt themes in beamer
\documentclass{beamer}

% Theme choice:
\usetheme{CambridgeUS}

% Title page details: 
\title{Papoulis Question 5.14} 
\author{Suryaansh Jain}
\date{\today}
\logo{\large \LaTeX{}}

\providecommand{\pr}[1]{\ensuremath{\Pr\left(#1\right)}}


\begin{document}

% Title page frame
\begin{frame}
    \titlepage 
\end{frame}

% Remove logo from the next slides
\logo{}


% Outline frame
\begin{frame}{Outline}
    \tableofcontents
\end{frame}


\section{Question}
\begin{frame}{Question}

$x$ and $y$ are independent Gamma random variables with  common parameters $\alpha$ and $\beta$. Find the p.d.f. of \\ 
(a) $x + y$. \\ 
(b) $x/y$. \\ 
(c) $x/(x +y)$.
\end{frame}

\section{Solution}
\begin{frame}{Genaral}
    \begin{equation}
        f(x) = 
        \begin{cases}
            \frac{x^{\alpha - 1}}{\Gamma (\alpha) {\beta}^{\alpha}} e^{-x/\beta}, & x \geq 0 \\
            0, & otherwise
        \end{cases}
    \end{equation}
        
    \begin{equation}
        f(y) = 
        \begin{cases}
            \frac{y^{\alpha - 1}}{\Gamma (\alpha) {\beta}^{\alpha}} e^{-y/\beta}, & y \geq 0 \\
            0, & otherwise
        \end{cases}
    \end{equation}
\end{frame} 

\begin{frame}{Genaral}
    \begin{align}
        \phi_x (\omega) = \frac{1}{(1 - j \omega \beta )^\alpha} \\
        \phi_y (\omega) = \frac{1}{(1 - j \omega \beta )^\alpha}
    \end{align}
\end{frame}

\begin{frame}{Part (a)}

    As $x$ and $y$ are independent $\phi_{x+y} (\omega) = \phi_x (\omega) \phi_y (\omega) $.

    \begin{align}
        \phi_{x+y} (\omega) &= \frac{1}{(1 - j \omega \beta )^{2\alpha}} \\
        \implies \phi_{x+y} (\omega) &\sim \text{Gamma} (2\alpha, \beta)
    \end{align}
    
\end{frame} 

\begin{frame}{Part (b)}
    Let $z =  x/y$ then we know,
    
    \begin{align}
        f_z(z) &= \int_{0}^{\infty} y\ f_{xy}(yz , y) \,dy \\
        \implies f_z(z) &= \int_{0}^{\infty} y\ \frac{(y^2 z)^{\alpha - 1}}{(\Gamma(\alpha) \beta ^ \alpha)^2} e^{-(1+z)y/\beta} \,dy\\
        &= \frac{z^{\alpha-1}}{\Gamma(\alpha) \beta ^ \alpha)^2}\int_{0}^{\infty} y^{(2\alpha - 1)}e^{-(1+z)y/\beta} \,dy\\
        &= \frac{z^{\alpha-1}}{\Gamma(\alpha) \beta ^ \alpha)^2} \frac{\beta^{2\alpha}}{(1+z)^{2\alpha}}\int_{0}^{\infty} u^{2\alpha - 1} e^{-u} \,du\\
        &= \frac{\Gamma(2\alpha)\ z^{\alpha - 1}}{(\Gamma(\alpha))^2 (1+z)^ {2\alpha}},\ \ \ \ z > 0
    \end{align}
\end{frame}

\begin{frame}{Part (c)}
    Let, 
    \begin{equation}
        w = \frac{x}{x+y} = \frac{z}{z+1}
    \end{equation}
    
    \begin{equation}
        F_w(w) = P\left(\frac{z}{1+z} \leq w\right) = P\left(\frac{w}{1-w} \geq z\right) = F_z\left(\frac{w}{1-w}\right)
    \end{equation}
    
    Differentiation the above eqn. we get, 
    \begin{align}
        f_w(w) &= \frac{1}{(1-w)^2} f_z\left(\frac{w}{1-w} \right) \\
        &= \frac{\Gamma(2\alpha}{(\Gamma(\alpha))^2} w^{\alpha - 1} (1-w)^{\alpha - 1}\\
        &\sim \text{Beta} (\alpha, \alpha)
    \end{align}
\end{frame}

\end{document}

