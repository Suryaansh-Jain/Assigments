%
% Welcome to Overleaf --- just edit your LaTeX on the left,
% and we'll compile it for you on the right. If you open the
% 'Share' menu, you can invite other users to edit at the same
% time. See www.overleaf.com/learn for more info. Enjoy!
%
%%%%%%%%%%%%%%%%%%%%%%%%%%%%%%%%%%%%%%%%%%%%%%%%%%%%%%%%%%%%%%%


% Inbuilt themes in beamer
\documentclass{beamer}

% Theme choice:
\usetheme{AnnArbor}

% Title page details: 
\title{Papoulis Question 8.14} 
\author{Suryaansh Jain}
\date{\today}
\logo{\large \LaTeX{}}

\providecommand{\pr}[1]{\ensuremath{\Pr\left(#1\right)}}


\begin{document}

% Title page frame
\begin{frame}
    \titlepage 
\end{frame}

% Remove logo from the next slides
\logo{}


% Outline frame
\begin{frame}{Outline}
    \tableofcontents
\end{frame}


\section{Question}
\begin{frame}{Question}

A coin is tossed once, and heads shows. Assuming that the probability $p$ of heads is the value of a random variable \textbf{p} uniformly distributed in the interval $(0.4, 0.6)$, find its bayesian estimate.
\end{frame}

\section{Solution}
\begin{frame}{Solution}
We know,

\begin{equation}
    f(p|M) = \frac{p^k q^{n-k} f(p)}{\int _{0} ^ {1} {p^k q^{n-k} f(p)dp}}
\end{equation}

In our case $k = 1$ we get,

\begin{equation}
    f(p) = \begin{cases}
            0.5, & 0.4 \leq p \leq 0.6 \\
            0, & otherwise \\
        \end{cases}
\end{equation}

\begin{equation}
    f(p|1) = \begin{cases}
            10p, & 0.4 \leq p \leq 0.6 \\
            0, & otherwise
        \end{cases}
\end{equation}

\begin{equation}
    \implies \hat{p} = 0.5067
\end{equation}
    
\end{frame}

\end{document}

