%
% Welcome to Overleaf --- just edit your LaTeX on the left,
% and we'll compile it for you on the right. If you open the
% 'Share' menu, you can invite other users to edit at the same
% time. See www.overleaf.com/learn for more info. Enjoy!
%
%%%%%%%%%%%%%%%%%%%%%%%%%%%%%%%%%%%%%%%%%%%%%%%%%%%%%%%%%%%%%%%


% Inbuilt themes in beamer
\documentclass{beamer}

% Theme choice:
\usetheme{CambridgeUS}

% Title page details: 
\title{12th Class NCERT Chapter 13 Exercise Question 6} 
\author{Suryaansh Jain}
\date{\today}
\logo{\large \LaTeX{}}

\providecommand{\pr}[1]{\ensuremath{\Pr\left(#1\right)}}


\begin{document}

% Title page frame
\begin{frame}
    \titlepage 
\end{frame}

% Remove logo from the next slides
\logo{}


% Outline frame
\begin{frame}{Outline}
    \tableofcontents
\end{frame}


\section{Question}
\begin{frame}{Question}

There are 3 coins, one is a two headed coin (has both sides head), one is a biased coin that shows head with $75\%$ probability and the last coin is unbiased. A coin is chosen at random and tossed. It shows head, what is the probability that it is a two headed coin?
\end{frame}


\section{Solution}
\begin{frame}{Solution}
Let us call the two headed coin $\mathcal{C}_1$, the biased coin $\mathcal{C}_2$ and the unbiased coin $\mathcal{C}_3$
\end{frame} 

\section{For $\mathcal{C}_1$}
\begin{frame}{For $\mathcal{C}_1$}
Let the random variable $X_{\mathcal{C}_1}$ denote what the coin shows. Then, we see that the sample space is $S = \cbrak{0, 1}$ where $1$ is head and $0$ is tail. The PMF is given by
\begin{equation}
\pr{X_{\mathcal{C}_1} = k} = 
\begin{cases}
1, & k = 1 \\
0, & \text{otherwise} 
\end{cases}
\label{pmf}
\end{equation}
\end{frame} 

\section{For $\mathcal{C}_2$}
\begin{frame}{For $\mathcal{C}_2$}
Let the random variable $X_{\mathcal{C}_2}$ denote what the coin shows. Then, we see that the sample space is $S = \cbrak{0, 1}$ where $1$ is head and $0$ is tail. The PMF is given by
\begin{equation}
\pr{X_{\mathcal{C}_2} = k} = 
\begin{cases}
\frac{3}{4}, & k = 1 \\
\frac{1}{4}, & k = 0 \\
0, & \text{otherwise}
\end{cases}
\label{pmf}
\end{equation}
\end{frame} 

\section{For $\mathcal{C}_3$}
\begin{frame}{For $\mathcal{C}_3$}
Let the random variable $X_{\mathcal{C}_3}$ denote what the coin shows. Then, we see that the sample space is $S = \cbrak{0, 1}$ where $1$ is head and $0$ is tail. The PMF is given by
\begin{equation}
\pr{X_{\mathcal{C}_3} = k} = 
\begin{cases}
\frac{1}{2}, & k = 1 \\
\frac{1}{2}, & k = 0 \\
0, & \text{otherwise}
\end{cases}
\label{pmf}
\end{equation}
\end{frame} 

\section{Tossing the coin}
\begin{frame} {Tossing the coin}
Let the random variable $X$ denote the coin we picked. Then we see that the same space is $S = \cbrak{1,2,3}$ where $1$ is $\mathcal{C}_1$, $2$ is $\mathcal{C}_2$ and $3$ is $\mathcal{C}_3$. The PMF is given by
 
\begin{equation}
\pr{X = k} = 
\begin{cases}
\frac{1}{3}, & 1 \leq k \leq 3 \\
0, & \text{otherwise}
\end{cases}
\label{pmf}
\end{equation}
\end{frame}

\section{Solving}
\begin{frame}{Solving}
    Given that the coin shows head we have to find the conditional probability that the coin is $\mathcal{C}_1$. This is given by
\begin{equation}
    \pr{X = 1 | K}
\end{equation}
Where $K$ is the condition that the coin shows a head.\newline 
Let $E$ be the event : A coin is chosen at random and is tossed, the outcome of this toss is a head and the coin is a two headed coin.
\end{frame}

\section{Solving (Contd.)}
\begin{frame}{Solving (Contd.)}
Now,
\begin{align}
   & \pr{E} = \frac{\pr{X = 1 \cap \mathcal{C}_1}}{\pr{K}} \\
    &\pr{X = 1 \cap \mathcal{C}_1} = \frac{1}{3} \\
    &\pr{K} = \sum_{i = 1}^{3}{\pr{X = i \cap \mathcal{C}_i}} \\
    &\implies \pr{K} = \frac{1}{3} + \frac{1}{4} + \frac{1}{6} = \frac{3}{4} \\
    &\implies \pr{E} = \frac{4}{9}
\end{align}
\end{frame}

\end{document}