\documentclass[journal,12pt,twocolumn]{IEEEtran}
%
\usepackage{setspace}
\usepackage{gensymb}
%\doublespacing
\singlespacing

%\usepackage{graphicx}
%\usepackage{amssymb}
%\usepackage{relsize}
\usepackage[cmex10]{amsmath}
%\usepackage{amsthm}
%\interdisplaylinepenalty=2500
%\savesymbol{iint}
%\usepackage{txfonts}
%\restoresymbol{TXF}{iint}
%\usepackage{wasysym}
\usepackage{amsthm}
%\usepackage{iithtlc}
\usepackage{mathrsfs}
\usepackage{txfonts}
\usepackage{stfloats}
\usepackage{bm}
\usepackage{cite}
\usepackage{cases}
\usepackage{subfig}
%\usepackage{xtab}
\usepackage{longtable}
\usepackage{multirow}
%\usepackage{algorithm}
%\usepackage{algpseudocode}
\usepackage{enumitem}
\usepackage{mathtools}
\usepackage{tikz}
\usepackage{circuitikz}
\usepackage{verbatim}
%\usepackage{tfrupee}
\usepackage[breaklinks=true]{hyperref}
%\usepackage{stmaryrd}
\usepackage{tkz-euclide} % loads  TikZ and tkz-base
%\usetkzobj{all}
\usepackage{listings}
    \usepackage{color}                                            %%
    \usepackage{array}                                            %%
    \usepackage{longtable}                                        %%
    \usepackage{calc}                                             %%
    \usepackage{multirow}                                         %%
    \usepackage{hhline}                                           %%
    \usepackage{ifthen}                                           %%
  %optionally (for landscape tables embedded in another document): %%
    \usepackage{lscape}     
\usepackage{multicol}
\usepackage{chngcntr}
%\usepackage{enumerate}

%\usepackage{wasysym}
%\newcounter{MYtempeqncnt}
\DeclareMathOperator*{\Res}{Res}
%\renewcommand{\baselinestretch}{2}
\renewcommand\thesection{\arabic{section}}
\renewcommand\thesubsection{\thesection.\arabic{subsection}}
\renewcommand\thesubsubsection{\thesubsection.\arabic{subsubsection}}

\renewcommand\thesectiondis{\arabic{section}}
\renewcommand\thesubsectiondis{\thesectiondis.\arabic{subsection}}
\renewcommand\thesubsubsectiondis{\thesubsectiondis.\arabic{subsubsection}}

% correct bad hyphenation here
\hyphenation{op-tical net-works semi-conduc-tor}
\def\inputGnumericTable{}                                 %%

\lstset{
%language=C,
frame=single, 
breaklines=true,
columns=fullflexible
}
%\lstset{
%language=tex,
%frame=single, 
%breaklines=true
%}

\begin{document}
\newtheorem{theorem}{Theorem}[section]
\newtheorem{problem}{Problem}
\newtheorem{proposition}{Proposition}[section]
\newtheorem{lemma}{Lemma}[section]
\newtheorem{corollary}[theorem]{Corollary}
\newtheorem{example}{Example}[section]
\newtheorem{definition}[problem]{Definition}
%\newtheorem{thm}{Theorem}[section] 
%\newtheorem{defn}[thm]{Definition}
%\newtheorem{algorithm}{Algorithm}[section]
%\newtheorem{cor}{Corollary}
\newcommand{\BEQA}{\begin{eqnarray}}
\newcommand{\EEQA}{\end{eqnarray}}
\newcommand{\define}{\stackrel{\triangle}{=}}
\bibliographystyle{IEEEtran}
%\bibliographystyle{ieeetr}
\providecommand{\mbf}{\mathbf}
\providecommand{\pr}[1]{\ensuremath{\Pr\left(#1\right)}}
\providecommand{\cdf}[2]{\ensuremath{F_{#1}\left(#2\right)}}
\providecommand{\qfunc}[1]{\ensuremath{Q\left(#1\right)}}
\providecommand{\sbrak}[1]{\ensuremath{{}\left[#1\right]}}
\providecommand{\lsbrak}[1]{\ensuremath{{}\left[#1\right.}}
\providecommand{\rsbrak}[1]{\ensuremath{{}\left.#1\right]}}
\providecommand{\brak}[1]{\ensuremath{\left(#1\right)}}
\providecommand{\lbrak}[1]{\ensuremath{\left(#1\right.}}
\providecommand{\rbrak}[1]{\ensuremath{\left.#1\right)}}
\providecommand{\cbrak}[1]{\ensuremath{\left\{#1\right\}}}
\providecommand{\lcbrak}[1]{\ensuremath{\left\{#1\right.}}
\providecommand{\rcbrak}[1]{\ensuremath{\left.#1\right\}}}
\theoremstyle{remark}
\newtheorem{rem}{Remark}
\newcommand{\sgn}{\mathop{\mathrm{sgn}}}
%\providecommand{\abs}[1]{\left\vert#1\right\vert}
\providecommand{\res}[1]{\Res\displaylimits_{#1}} 
%\providecommand{\norm}[1]{\left\lVert#1\right\rVert}
%\providecommand{\norm}[1]{\lVert#1\rVert}
\providecommand{\mtx}[1]{\mathbf{#1}}
%\providecommand{\mean}[1]{E\left[ #1 \right]}
\providecommand{\fourier}{\overset{\mathcal{F}}{ \rightleftharpoons}}
%\providecommand{\hilbert}{\overset{\mathcal{H}}{ \rightleftharpoons}}
\providecommand{\system}{\overset{\mathcal{H}}{ \longleftrightarrow}}
	%\newcommand{\solution}[2]{\textbf{Solution:}{#1}}
\newcommand{\solution}{\noindent \textbf{Solution: }}
\newcommand{\cosec}{\,\text{cosec}\,}
\providecommand{\dec}[2]{\ensuremath{\overset{#1}{\underset{#2}{\gtrless}}}}
\newcommand{\myvec}[1]{\ensuremath{\begin{pmatrix}#1\end{pmatrix}}}
\newcommand{\mydet}[1]{\ensuremath{\begin{vmatrix}#1\end{vmatrix}}}

%\numberwithin{equation}{section}
%\numberwithin{equation}{subsection}
%\numberwithin{problem}{section}
%\numberwithin{definition}{section}
\makeatletter
\@addtoreset{figure}{problem}
\makeatother
\let\StandardTheFigure\thefigure
\let\vec\mathbf
%\renewcommand{\thefigure}{\theproblem.\arabic{figure}}
\renewcommand{\thefigure}{\arabic{figure}}
%\setlist[enumerate,1]{before=\renewcommand\theequation{\theenumi.\arabic{equation}}
%\counterwithin{equation}{enumi}
%\renewcommand{\theequation}{\arabic{subsection}.\arabic{equation}}
\def\putbox#1#2#3{\makebox[0in][l]{\makebox[#1][l]{}\raisebox{\baselineskip}[0in][0in]{\raisebox{#2}[0in][0in]{#3}}}}
     \def\rightbox#1{\makebox[0in][r]{#1}}
     \def\centbox#1{\makebox[0in]{#1}}
     \def\topbox#1{\raisebox{-\baselineskip}[0in][0in]{#1}}
     \def\midbox#1{\raisebox{-0.5\baselineskip}[0in][0in]{#1}}
\title{Assignment 3 (NCERT Class 12 Probability)}
\author{Suryaansh Jain (CS21BTECH11057)}	
\maketitle

\begin{abstract}
This document contains the solution to Question 6 of Exercise 3 of Chapter 13 (Probability) in the NCERT Class 12 Textbook. 
\end{abstract}

\noindent \textbf{Example 3.} There are 3 coins, one is a two headed coin (has both sides head), one is a biased coin that shows head with $75\%$ probability and the last coin is unbiased. A coin is chosen at random and tossed. It shows head, what is the probability that it is a two headed coin? \newline

 \solution Let us call the two headed coin $\mathcal{C}_1$, the biased coin $\mathcal{C}_2$ and the unbiased coin $\mathcal{C}_3$. \newline


\noindent For $\mathcal{C}_1$ :
\newline Let the random variable $X_{\mathcal{C}_1}$ denote what the coin shows. Then, we see that the sample space is $S = \cbrak{0, 1}$ where $1$ is head and $0$ is tail. The PMF is given by

\begin{equation}
\pr{X_{\mathcal{C}_1} = k} = 
\begin{cases}
1, & k = 1 \\
0, & \text{otherwise} 
\end{cases}
\label{pmf}
\end{equation}


\noindent For $\mathcal{C}_2$ :
\newline Let the random variable $X_{\mathcal{C}_2}$ denote what the coin shows. Then, we see that the sample space is $S = \cbrak{0, 1}$ where $1$ is head and $0$ is tail. The PMF is given by

\begin{equation}
\pr{X_{\mathcal{C}_2} = k} = 
\begin{cases}
\frac{3}{4}, & k = 1 \\
\frac{1}{4}, & k = 0 \\
0, & \text{otherwise}
\end{cases}
\label{pmf}
\end{equation}


\noindent For $\mathcal{C}_3$ :
\newline Let the random variable $X_{\mathcal{C}_3}$ denote what the coin shows. Then, we see that the sample space is $S = \cbrak{0, 1}$ where $1$ is head and $0$ is tail. The PMF is given by

\begin{equation}
\pr{X_{\mathcal{C}_3} = k} = 
\begin{cases}
\frac{1}{2}, & k = 1 \\
\frac{1}{2}, & k = 0 \\
0, & \text{otherwise}
\end{cases}
\label{pmf}
\end{equation}

 \noindent Let the random variable $X$ denote the coin we picked. Then we see that the same space is $S = \cbrak{1,2,3}$ where $1$ is $\mathcal{C}_1$, $2$ is $\mathcal{C}_2$ and $3$ is $\mathcal{C}_3$. The PMF is given by
 
\begin{equation}
\pr{X = k} = 
\begin{cases}
\frac{1}{3}, & 1 \leq k \leq 3 \\
0, & \text{otherwise}
\end{cases}
\label{pmf}
\end{equation}

\noindent Given that the coin shows head we have to find the conditional probability that the coin is $\mathcal{C}_1$. This is given by

\begin{equation}
    \pr{X = 1 | K}
\end{equation}

\noindent Where $K$ is the condition that the coin shows a head.\newline 

\noindent Let $E$ be the event : A coin is chosen at random and is tossed, the outcome of this toss is a head. The coin is a two headed coin.

\noindent Now,

\begin{align}
   & \pr{E} = \frac{\pr{X = 1 \cap \mathcal{C}_1}}{\pr{K}} \\
    &\pr{X = 1 \cap \mathcal{C}_1} = \frac{1}{3} \\
    &\pr{K} = \sum_{i = 1}^{3}{\pr{X = i \cap \mathcal{C}_i}} \\
    &\impliespr{K} = \frac{1}{3} + \frac{1}{4} + \frac{1}{6} = \frac{3}{4} \\
    &\implies \pr{E} = \frac{4}{9}
\end{align}

\end{document}
