%
% Welcome to Overleaf --- just edit your LaTeX on the left,
% and we'll compile it for you on the right. If you open the
% 'Share' menu, you can invite other users to edit at the same
% time. See www.overleaf.com/learn for more info. Enjoy!
%
%%%%%%%%%%%%%%%%%%%%%%%%%%%%%%%%%%%%%%%%%%%%%%%%%%%%%%%%%%%%%%%


% Inbuilt themes in beamer
\documentclass{beamer}

% Theme choice:
\usetheme{CambridgeUS}

% Title page details: 
\title{Papoulis Question 5.14} 
\author{Suryaansh Jain}
\date{\today}
\logo{\large \LaTeX{}}

\providecommand{\pr}[1]{\ensuremath{\Pr\left(#1\right)}}


\begin{document}

% Title page frame
\begin{frame}
    \titlepage 
\end{frame}

% Remove logo from the next slides
\logo{}


% Outline frame
\begin{frame}{Outline}
    \tableofcontents
\end{frame}


\section{Question}
\begin{frame}{Question}

Given that random variable x is of continuous type. we form the random variable $y = g(x)$.
(a) Find $f_y(y)$ if $g(x) = 2F_x(x) +4$. (b) Find $g(x)$ such that y is uniform in the interval
(8, 10)
\end{frame}


\section{Solution}
\begin{frame}{Solution}
\begin{align}
    g(x) = 2F_x(x) + 4 \\ 
    g'(x) = 2f(x) \\
\end{align}

If $4 < x < 6$ then $y = 2F_x(x) + 4$ has a unique solution $x_1$ and 
\begin{equation}
    f_f(y) = \frac{f_x(x_1)}{2f_x(x_1)} = 0.5
\end{equation}

similarly $g(x) = 2F_x(x) + 4$
\end{frame} 

\end{document}
