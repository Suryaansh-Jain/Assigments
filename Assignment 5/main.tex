%
% Welcome to Overleaf --- just edit your LaTeX on the left,
% and we'll compile it for you on the right. If you open the
% 'Share' menu, you can invite other users to edit at the same
% time. See www.overleaf.com/learn for more info. Enjoy!
%
%%%%%%%%%%%%%%%%%%%%%%%%%%%%%%%%%%%%%%%%%%%%%%%%%%%%%%%%%%%%%%%


% Inbuilt themes in beamer
\documentclass{beamer}

% Theme choice:
\usetheme{CambridgeUS}

% Title page details: 
\title{Papoulis Question 2.23} 
\author{Suryaansh Jain}
\date{\today}
\logo{\large \LaTeX{}}

\providecommand{\pr}[1]{\ensuremath{\Pr\left(#1\right)}}


\begin{document}

% Title page frame
\begin{frame}
    \titlepage 
\end{frame}

% Remove logo from the next slides
\logo{}


% Outline frame
\begin{frame}{Outline}
    \tableofcontents
\end{frame}


\section{Question}
\begin{frame}{Question}

Box 1 contains 1 white and 999 red balls. Box 2 contains 1 red and 999 white balls, A ball
is picked from a randomly selected box. If the ball is red what is the probability that it came from box 1
\end{frame}


\section{Solution}
\begin{frame}{Solution}
Let us call the first box $\mathcal{B}_1$ and the second box $\mathcal{B}_2$
\end{frame} 

\section{For $\mathcal{B}_1$}
\begin{frame}{For $\mathcal{B}_1$}
Let the random variable $X_{\mathcal{B}_1}$ denote the colour of the ball picked. Then, we see that the sample space is $S = {0, 1}$ where $1$ is red and $0$ is white. The PMF is given by
\begin{equation}
\pr{X_{\mathcal{C}_1} = k} = 
\begin{cases}
\frac{999}{1000}, & k = 1 \\
\frac{1}{1000}, & k = 0 \\
0, & \text{otherwise} 
\end{cases}
\label{pmf}
\end{equation}
\end{frame} 

\section{For $\mathcal{B}_2$}
\begin{frame}{For $\mathcal{B}_2$}
Let the random variable $X_{\mathcal{B}_2}$ denote the colour of the ball picked. Then, we see that the sample space is $S = {0, 1}$ where $1$ is red and $0$ is white. The PMF is given by
\begin{equation}
\pr{X_{\mathcal{C}_2} = k} = 
\begin{cases}
\frac{999}{1000}, & k = 0 \\
\frac{1}{1000}, & k = 1 \\
0, & \text{otherwise} 
\end{cases}
\label{pmf}
\end{equation}
\end{frame} 

\section{Picking a box}
\begin{frame} {Picking a box}
Let the random variable $X$ denote the box we picked. Then we see that the same space is $S = \cbrak{1,2}$ where $1$ is $\mathcal{B}_1$ and $2$ is $\mathcal{B}_2$. The PMF is given by
 
\begin{equation}
\pr{X = k} = 
\begin{cases}
\frac{1}{2}, & 1 \leq k \leq 2 \\
0, & \text{otherwise}
\end{cases}
\label{pmf}
\end{equation}
\end{frame}

\section{Solving}
\begin{frame}{Solving}
    Given that the ball is red we have to find the conditional probability that the box is $\mathcal{B}_1$. This is given by
\begin{equation}
    \pr{X = 1 | K}
\end{equation}
Where $K$ is the condition that the ball is red.\newline 
Let $E$ be the event : A box is chosen at random and a ball is picked, the ball is red and it is from box 1.
\end{frame}

\section{Solving (Contd.)}
\begin{frame}{Solving (Contd.)}
Now,
\begin{align}
   & \pr{E} = \frac{\pr{X = 1 , \mathcal{B}_1}}{\pr{K}} \\
    &\pr{X = 1 , \mathcal{B}_1} = \frac{999}{2000} \\
    &\pr{K} = \sum_{i = 1}^{2}{\pr{X = i , \mathcal{B}_i}} \\
    &\implies \pr{K} = \frac{999}{2000} + \frac{1}{2000} = \frac{1}{2} \\
    &\implies \pr{E} = \frac{999}{1000}
\end{align}
\end{frame}

\end{document}

